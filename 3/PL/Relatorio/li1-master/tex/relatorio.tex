\documentclass[a4paper]{article}

\usepackage[utf8]{inputenc}
\usepackage[portuguese]{babel}
\usepackage{a4wide}

\title{Projeto de Laboratórios de Informática 1\\Grupo 107}
\author{Eduardo Rocha (A77048) \and André Vieira (A78322)}
\date{\today}

\begin{document}

\maketitle

\begin{abstract}
  Este documento apresenta o relatório do projeto de Laboratórios 
  de Informática 1, da Licenciatura em Engenharia Informática da 
  Universidade do Minho.

  O trabalho feito corresponde às tarefas 1, 2, 3 e 4, que são 
  responsáveis por construir um tabuleiro de jogo, realizar 
  jogadas, codificar/descodificar o tabuleiro de jogo e realizar
  a passagem de tempo, respetivamente.
\end{abstract}

\tableofcontents

\section{Introdução}
\label{sec:intro}
  
Neste projecto foi-nos proposta a realização de um jogo de Bomberman
feito todo em Haskell. No nosso trabalho foram realizadas as Tarefas 
1, 2, 3 e 4. Esta secção apresenta uma introdução ao trabalho realizado
em cada tarefa.

Na \emph{Tarefa 1} é apresentado o problema de construir um mapa válido 
de Bomberman. O tabuleiro de jogo deve obedecer a algumas regras: a
altura deve ser igual ao comprimento, o comprimento do mapa deve ser 
maior que cinco e o comprimento deve ser sempre ímpar. Após o mapa 
for produzido, é anexado ao mapa uma lista de strings correspondentes
à localização de jogadores, power ups e bombas.

Na \emph{Tarefa 2} é pedida a implementação de comandos de jogo, ou seja:
U, D, L, R e B, representando estes comandos a movimentação de um jogador
para cima, baixo, esquerda e direita, e a colocação de uma bomba. 

Na \emph{Tarefa 3} é realizada a compressão e descompressão de informação 
correspondente a um tabuleiro de jogo. Nesta tarefa, durante a comprressão,
tenta-se reduzir o tamanho do tabuleiro ao mínimo tamanho possível e 
durante a descompressão tenta-se restaurar o tabuleiro comprimido para a 
sua forma original.

Na \emph{Tarefa 4} realizamos a passagem de tempo dentro do jogo de Bomberman.
A passagem de tempo inclui a redução do temporizador das bombas assim como a 
sua explosão caso o temporizador seja reduzido para zero. No caso 
da explosão de uma bomba, os jogadores, power ups e primeiros tijolos que 
se encontrem dentro do raio da explosão devem ser retirados. Caso o 
tempo restante chegue a um certo ponto predefinido, relacionado com 
a dimensão do tabuleiro de jogo, o mapa começa a fechar-se, ou seja,
são colocadas pedras numa forma de espiral em direção dos ponteiros do
relógio em todos os instantes de jogo, até que o jogo acabe. Caso uma 
pedra seja colocada em cima de um jogador, power up ou bomba, a informação
correspondente a essa identidade será removida.

Nas secções seguintes serão descritos mais detalhes acerca do trabalho. Na
secção~\ref{sec:problemas} serão descritos os problemas encontrados durante 
a realização de cada uma das tarefas, na secção~\ref{sec:solucao} 
desenvolvemos os métodos usados para resolver os problemas encontrados e 
na secção~\ref{sec:conclusao} é apresentada a conclusão do relatório.

\section{Descrição do Problema}
\label{sec:problema}

A \emph{Tarefa 1} serve para gerar um mapa válido de Bomberman. Surgem então
dois problemas: o primeiro que inclui a criação de um tabuleiro de
jogo vazio e o segundo que inclui a geração da lista de strings 
correspondente às descrições das posições e tipos de power ups.
Pretendemos então criar o tabuleiro e a lista de informações separadamente,
e depois comparar a lista ao mapa para poder colocar os tijolos e criar 
as informações sobre power ups, juntando as duas partes no final.

A \emph{Tarefa 2} tem a função de realizar os comandos feitos pelos jogadores, 
que podem incluir a movimentação de um jogador ou a colocação de uma 
bomba. Se receber os comandos U, D, L ou R a função deverá verificar 
se existem objetos nas novas coordenadas pretendidas que impedem o 
movimento e caso existam, não deve executar quaisquer mudanças.
Caso seja dado o comando B, deve verificar se é possível colocar 
uma bomba naquelas coordenadas e caso seja, deverá incluir a informação 
necessária sobre a posição da bomba, raio, temporizador e qual jogador 
a colocou. Quando um jogador se deslocar para as mesmas coordenadas onde
se encontra um power up, a tarefa também deverá remover a informação 
correspondente ao power up do mapa de jogo e deverá acrescentar à linha
do jogador que este apanhou o power up.

A \emph{Tarefa 3} apresenta o problema de codificar e descodificar o mapa 
de jogo. Durante o processo de codificação o programa deverá tentar 
reduzir para o mínimo tamanho possível o mapa de jogo, sem sacrificar 
qualquer tipo de informação necessária para o jogo. No processo de 
descodificação, o mapa de jogo codificado deve ser restaurado para 
o seu estado original.

A \emph{Tarefa 4} trata da passagem de tempo durante o jogo de Bomberman,
criando três problemas: o primeiro que é a redução do temporizador 
de todas as bombas, o segundo que é verificar se alguma das bombas 
reduzidas tem agora um temporizador de valor zero e nesse caso 
deverá executar as mudanças necessárias ao tabuleiro por causa 
da explosão da bomba e o terceiro problema que envolve fechar o 
mapa quando o tempo de jogo excede o limite predefinido, colocando 
uma pedra onde adequado.

\section{Concepção da Solução}
\label{sec:solucao}

As subsecções seguintes apresentarão as estruturas de dados usadas, 
a implementação usada e os testes realizados, respetivamente, para 
todas as tarefas realizadas.

\subsection{Estruturas de Dados}

Em todas as tarefas foram utilizadas principalmente listas. Foram 
usadas listas de Strings para as várias representações do mapa em 
diferentes estados do trabalho e listas de Integers na \emph{Tarefa 1}
para a lista de números gerados aleatoriamente pela tarefa para 
depois gerar todos os power ups. 

\subsection{Implementação}

A \emph{Tarefa 1} começa por verificar que as medidas fornecidas 
para o comprimento são válidas e depois chama as funções necessárias 
para construir o mapa se os inputs fornecidos forem válidos. O 
principal problema apresentado nesta tarefa foi descobrir como 
criar o mapa e criar a lista de números a usar para gerar a lista
de informação sobre os power ups e como comparar as duas listas 
para saber onde seriam colocados os power ups e os tijolos.
A solução encontrada foi criar uma versão do mapa com os tijolos 
já colocados e depois usar duas funções: a addBombas e a addFlames 
para criar as listas de power ups Bombas e Flames que seriam adicionados
ao mapa, e depois no final simplesmente juntar tudo - primeiro o mapa, 
depois as Bombas e depois os Flames.
\begin{verbatim}
mapaAux :: Int -> [String] -> [Int] -> [String] 
mapaAux l m p = m ++ addBombas m p 0 l ++ addFlames m p 0 l
\end{verbatim}


A \emph{Tarefa 2} primeiro verifica se os inputs fornecidos são válidos
e depois qual o tipo de input fornecido para saber quais funções deverá
chamar. Caso o input seja \emph{U}, \emph{D}, \emph{L} ou \emph{R}, o
objetivo será tentar mover o jogador que está a fazer a jogada para 
uma nova posição. Caso o input seja B, o objetivo será tentar colocar 
uma bomba na posição atual do jogador. Durante esta Tarefa, foi preciso 
atenção à interação entre vários elementos do jogo, nomeadamente quais 
podem coexistir na mesma posição e quais não podem. Um dos problemas 
principais que encontramos foi a implementação do que acontece quando
um jogador apanha um power up. Nesse caso, a informação correspondente
ao power up apanhado deverá ser eliminada e o um símbolo correspondente 
ao tipo de power up apanhado será adicionado à descrição do jogador. 
O principal problema aqui foi saber distinguir quando o power up 
apanhado foi o primeiro apanhado. A função responsável por atualizar 
a descrição do jogador é a função atualiza.
\begin{verbatim}
atualiza :: Int -> (Int,Int,String) -> Char -> String
atualiza a (x,y,s) c | c == '+' || c == '!' =  show a ++ [' '] ++ show x ++ [' '] 
                                   ++ show y ++ [' '] ++ s ++ [c]  
                     | otherwise = show a ++ [' '] ++ show x ++ [' '] ++ show y 
                                   ++ " " ++ s
\end{verbatim}

A \emph{Tarefa 3} está dividida em duas partes. A primeira parte inclui
a codificação do código e a segunda a descodificação do código. 
Durante a codificação do código são retiradas as paredes laterais
ao mapa e durante a descodificação sao restauradas as paredes removidas
durante a codificação. Para isto são utilizadas as funções compress 
e decompress, respetivamente.
\begin{verbatim}
compress :: [String] -> [String]
compress (h:t) = retiraP t

decompress :: [String] -> [String]
decompress l = linhaPU ((length h)+2) : adicionaH (adicionaV (h:t)) ((length h)+2) 
\end{verbatim}


A \emph{Tarefa 4} trata da passagem de tempo no jogo. Primeiro, a
tarefa tratará de reduzir o temporizador de todas as bombas. A seguir, 
irá verificar se existem bombas cujo temporizador atingiu zero. Por 
cada bomba prestes a explodir (temporizador a zero) chamará funções 
que estarão encarregues de alterar a informação do mapa afetada pela 
explosão da bomba, ou seja, tijolos, jogadores, power ups ou bombas 
que serão removidas/alteradas. Caso necessário, após serem realizadas
as alterações anteriormente descritas, o mapa continuará o seu processo
de ser fechado, ou seja, será colocada uma pedra (carácter cardinal)
no local adequado. Descobrir o local adequado foi o principal problema 
apresentado durante a realização desta tarefa. A solução encontrada foi
passar pelo mapa na direção dos ponteiros do relógio em forma de espiral,
removendo camadas exteriores preenchidas completamente por pedras. 
A função responsável pelo processo anteriormente descrito, a spiral, 
verifica a primeira linha para ver se é feita toda de pedras, depois verifica
a última coluna, a seguir verifica a última linha e finalmente a primeira 
coluna. Se verificar que existem espaços numa das linhas/colunas analizadas,
irá tentar colocar uma pedra no sítio adequado; se verificar que a
camada exterior do mapa está toda preenchida, irá remover essa camada, 
reiniciando a função para o mapa interior. Após a função acabar, 
o mapa será restaurado ao seu estado original.

\subsection{Testes}

Para testar as várias tarefas recorremos à plataforma disponibilizada 
pelos docentes assim como testes realizados manualmente por nós 
no terminal a cada tarefa. Como exemplo de mapa usamos também o 
mapa seguinte:
\begin{verbatim}
#########
#       #
# #?# # #
#     ? #
#?# # #?#
# ?  ?  #
# #?#?# #
#  ??   #
#########
+ 3 3
! 5 5
* 7 7 1 1 10
0 4 3 +
1 7 7
\end{verbatim}

\section{Conclusões}
\label{sec:conclusao}

Em conclusão, estamos contentes com os trabalhos realizados nas Tarefas
1, 2 e 4, mas achamos que a Tarefa 3 podería estar muito mais desenvolvida, 
estando bastante incompleta por falta de tempo. Gostariamos também de 
ter realizado os trabalhos necessários para as Tarefas 5 e 6, que não 
foram desenvolvidas pelo nosso grupo.

\end{document}